%% start of file `template.tex'.
%% Copyright 2006-2013 Xavier Danaux (xdanaux@gmail.com).
%
% This work may be distributed and/or modified under the
% conditions of the LaTeX Project Public License version 1.3c,
% available at http://www.latex-project.org/lppl/.


\documentclass[11pt,a4paper,sans]{moderncv}        % possible options include font size ('10pt', '11pt' and '12pt'), paper size ('a4paper', 'letterpaper', 'a5paper', 'legalpaper', 'executivepaper' and 'landscape') and font family ('sans' and 'roman')

% modern themes
\moderncvstyle{banking}                            % style options are 'casual' (default), 'classic', 'oldstyle' and 'banking'
\moderncvcolor{blue}                                % color options 'blue' (default), 'orange', 'green', 'red', 'purple', 'grey' and 'black'
%\renewcommand{\familydefault}{\sfdefault}         % to set the default font; use '\sfdefault' for the default sans serif font, '\rmdefault' for the default roman one, or any tex font name
\nopagenumbers{}                                  % uncomment to suppress automatic page numbering for CVs longer than one page

% character encoding
\usepackage[utf8]{inputenc}
\usepackage{fontawesome}
\usepackage{tabularx}
\usepackage{ragged2e}
% if you are not using xelatex ou lualatex, replace by the encoding you are using
%\usepackage{CJKutf8}                              % if you need to use CJK to typeset your resume in Chinese, Japanese or Korean

% adjust the page margins
\usepackage[scale=0.87]{geometry}
\usepackage{multicol}
%\setlength{\hintscolumnwidth}{3cm}                % if you want to change the width of the column with the dates
%\setlength{\makecvtitlenamewidth}{10cm}           % for the 'classic' style, if you want to force the width allocated to your name and avoid line breaks. be careful though, the length is normally calculated to avoid any overlap with your personal info; use this at your own typographical risks...

\usepackage{import}

% personal data
\name{Artem}{Aliev}
% \title{Curriculum Vitae}                               % optional, remove / comment the line if not wanted
\address{}{}{}% optional, remove / comment the line if not wanted; the "postcode city" and and "country" arguments can be omitted or provided empty
% \phone[mobile]{909-839-3097}                   % optional, remove / comment the line if not wanted
% \phone[fixed]{01234 123456}                    % optional, remove / comment the line if not wanted
%\phone[fax]{+3~(456)~789~012}                      % optional, remove / comment the line if not wanted
% \email{xpan1@swarthmore.edu}                               % optional, remove / comment the line if not wanted
% \homepage{shawnpan.me}                         % optional, remove / comment the line if not wanted
% \extrainfo{}                 % optional, remove / comment the line if not wanted
%\photo[64pt][0.4pt]{picture}                       % optional, remove / comment the line if not wanted; '64pt' is the height the picture must be resized to, 0.4pt is the thickness of the frame around it (put it to 0pt for no frame) and 'picture' is the name of the picture file
%\quote{Some quote}                                 % optional, remove / comment the line if not wanted

% to show numerical labels in the bibliography (default is to show no labels); only useful if you make citations in your resume
%\makeatletter
%\renewcommand*{\bibliographyitemlabel}{\@biblabel{\arabic{enumiv}}}
%\makeatother
%\renewcommand*{\bibliographyitemlabel}{[\arabic{enumiv}]}% CONSIDER REPLACING THE ABOVE BY THIS

% bibliography with mutiple entries
%\usepackage{multibib}
%\newcites{book,misc}{{Books},{Others}}
  
\newcommand*{\customcventry}[7][.25em]{
  \begin{tabular}{@{}l} 
    {\bfseries #4}
  \end{tabular}
  \hfill% move it to the right
  \begin{tabular}{l@{}}
     {\bfseries #5}
  \end{tabular} \\
  \begin{tabular}{@{}l} 
    {\itshape #3}
  \end{tabular}
  \hfill% move it to the right
  \begin{tabular}{l@{}}
     {\itshape #2}
  \end{tabular}
  \ifx&#7&%
  \else{\\%
    \begin{minipage}{\maincolumnwidth}%
      \small#7%
    \end{minipage}}\fi%
  \par\addvspace{#1}}

\newcommand*{\customcvproject}[4][.25em]{
%   \vfill\noindent
  \begin{tabular}{@{}l} 
    {\bfseries #2}
  \end{tabular}
  \hfill% move it to the right
  \begin{tabular}{l@{}}
     {\itshape #3}
  \end{tabular}
  \ifx&#4&%
  \else{\\%
    \begin{minipage}{\maincolumnwidth}%
      \small#4%
    \end{minipage}}\fi%
  \par\addvspace{#1}}

\setlength{\tabcolsep}{12pt}

%----------------------------------------------------------------------------------
%            content
%----------------------------------------------------------------------------------
\begin{document}
%\begin{CJK*}{UTF8}{gbsn}                          % to typeset your resume in Chinese using CJK
%-----       resume       ---------------------------------------------------------
\makecvtitle
\vspace*{-23mm}

\begin{center}
\begin{tabular}{ c c c c }
\faEnvelopeO\enspace alievae0410@gmail.com & \faGithub\enspace \href{https://github.com/alievgithub}{alievgithub} &  \faMobile\enspace +7-964-585-6317

\end{tabular}
\end{center}

\section{EDUCATION}
{\customcventry{2019 - 2023}{B.S., System Analysis and Management (GPA: 4.3/5.0)}{Moscow Institute of Physics and Technology}{Moscow, Russia}{}{}}
{\customcventry{2019 - 2023}{B.S., Economics (GPA: 4.4/5.0)}{The Russian Presidential Academy of National Economy and Public Administration}{Moscow, Russia}{}{}}


\section{EXPERIENCE}

{\customcventry{April 2021 - July 2021}{HR Compensation \& Benefits Full-Time Intern}{Sanofi}{Moscow, Russia}{}
{\begin{itemize}
\item Built various types of dashboards in Excel 
\item Analyzed and calculated various big data \\such as turnover, bonuses or compensations
\item Participated in a tender with various insurance companies
\item Translated various presentations and communications
\end{itemize}
}
}

{\customcventry{September 2020 - November 2020}{Nanobiotechnology Laboratory Free-Time Intern}{MIPT}{Moscow, Russia}{тма}
{\begin{itemize}
\item Worked on 3D scanning of experimental mice
\end{itemize}
}
}

\section{PROJECTS AND COURSES}
{\customcvproject{Coursera.org}{2020 - 2021}
  {\begin{itemize}
    \item C++: The White Belt by Yandex
    \item Excel Skills for Business by Macquarie University
    \item New Technologies for Business Leaders by Rutgers University
  \end{itemize}
  }
}

{\customcvproject{Qmarketing Academy}{In Process}
  {\begin{itemize}
    \item Full immersion in marketing
  \end{itemize}
  }
}

{\customcvproject{University Team Projects}{2021}
  {\begin{itemize}
  \item \href{https://github.com/alievgithub/aliev_github/blob/master/StaveBot/StaveBot.py}{Write TG Bot for decomposing a music track into notes}
  \item \href{https://github.com/alievgithub/image_effects_bot}{Write code for adding effects to photos}
  \item \href{https://github.com/alievgithub/SpeechRecognitionBot}{Worked on speech recognition for music track search}
  \end{itemize}
  }
}


{\section{ACHIEVEMENTS}

{\customcvproject{Scholarships \& Awards}{2015 - 2019}
  {\begin{itemize}
     \item Prize-winner of the list of Russian and International Olympiads in mathematics, physics \& robotics
     \item Won a prize in several chess tournaments
  \end{itemize}
  }
}

{\customcvproject{Driving Licence}{2020}
  {\begin{itemize}
    \item Successfully passed the exam and received a license of category B, B1, M
  \end{itemize}
  }
}

{\customcvproject{Sports}{}
  {\begin{itemize}
    \item Kitesurfing and Windsurfing
    \item Tennis
    \item Aircraft control Yak-52
    \item Skydiving
  \end{itemize}
  }
}

}

% Publications from a BibTeX file without multibib
%  for numerical labels: \renewcommand{\bibliographyitemlabel}{\@biblabel{\arabic{enumiv}}}% CONSIDER MERGING WITH PREAMBLE PART
%  to redefine the heading string ("Publications"): \renewcommand{\refname}{Articles}

% Publications from a BibTeX file using the multibib package
%\section{Publications}
%\nocitebook{book1,book2}
%\bibliographystylebook{plain}
%\bibliographybook{publications}                   % 'publications' is the name of a BibTeX file
%\nocitemisc{misc1,misc2,misc3}
%\bibliographystylemisc{plain}
%\bibliographymisc{publications}                   % 'publications' is the name of a BibTeX file

%-----       letter       ---------------------------------------------------------

\end{document}


%% end of file `template.tex'.

