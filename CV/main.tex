%% start of file `template.tex'.
%% Copyright 2006-2013 Xavier Danaux (xdanaux@gmail.com).
%
% This work may be distributed and/or modified under the
% conditions of the LaTeX Project Public License version 1.3c,
% available at http://www.latex-project.org/lppl/.


\documentclass[11pt,a4paper,sans]{moderncv}        % possible options include font size ('10pt', '11pt' and '12pt'), paper size ('a4paper', 'letterpaper', 'a5paper', 'legalpaper', 'executivepaper' and 'landscape') and font family ('sans' and 'roman')

% modern themes
\moderncvstyle{banking}                            % style options are 'casual' (default), 'classic', 'oldstyle' and 'banking'
\moderncvcolor{blue}                                % color options 'blue' (default), 'orange', 'green', 'red', 'purple', 'grey' and 'black'
%\renewcommand{\familydefault}{\sfdefault}         % to set the default font; use '\sfdefault' for the default sans serif font, '\rmdefault' for the default roman one, or any tex font name
\nopagenumbers{}                                  % uncomment to suppress automatic page numbering for CVs longer than one page

% character encoding
\usepackage[utf8]{inputenc}
\usepackage{fontawesome}
\usepackage{tabularx}
\usepackage{ragged2e}
% if you are not using xelatex ou lualatex, replace by the encoding you are using
%\usepackage{CJKutf8}                              % if you need to use CJK to typeset your resume in Chinese, Japanese or Korean

% adjust the page margins
\usepackage[scale=0.89]{geometry}
\usepackage{multicol}
%\setlength{\hintscolumnwidth}{3cm}                % if you want to change the width of the column with the dates
%\setlength{\makecvtitlenamewidth}{10cm}           % for the 'classic' style, if you want to force the width allocated to your name and avoid line breaks. be careful though, the length is normally calculated to avoid any overlap with your personal info; use this at your own typographical risks...

\usepackage{import}

% personal data
\name{Artem}{Aliev}
% \title{Curriculum Vitae}                               % optional, remove / comment the line if not wanted
\address{}{}{}% optional, remove / comment the line if not wanted; the "postcode city" and and "country" arguments can be omitted or provided empty
% \phone[mobile]{909-839-3097}                   % optional, remove / comment the line if not wanted
% \phone[fixed]{01234 123456}                    % optional, remove / comment the line if not wanted
%\phone[fax]{+3~(456)~789~012}                      % optional, remove / comment the line if not wanted
% \email{xpan1@swarthmore.edu}                               % optional, remove / comment the line if not wanted
% \homepage{shawnpan.me}                         % optional, remove / comment the line if not wanted
% \extrainfo{}                 % optional, remove / comment the line if not wanted
%\photo[64pt][0.4pt]{picture}                       % optional, remove / comment the line if not wanted; '64pt' is the height the picture must be resized to, 0.4pt is the thickness of the frame around it (put it to 0pt for no frame) and 'picture' is the name of the picture file
%\quote{Some quote}                                 % optional, remove / comment the line if not wanted

% to show numerical labels in the bibliography (default is to show no labels); only useful if you make citations in your resume
%\makeatletter
%\renewcommand*{\bibliographyitemlabel}{\@biblabel{\arabic{enumiv}}}
%\makeatother
%\renewcommand*{\bibliographyitemlabel}{[\arabic{enumiv}]}% CONSIDER REPLACING THE ABOVE BY THIS

% bibliography with mutiple entries
%\usepackage{multibib}
%\newcites{book,misc}{{Books},{Others}}
  
\newcommand*{\customcventry}[7][.25em]{
  \begin{tabular}{@{}l} 
    {\bfseries #4}
  \end{tabular}
  \hfill% move it to the right
  \begin{tabular}{l@{}}
     {\bfseries #5}
  \end{tabular} \\
  \begin{tabular}{@{}l} 
    {\itshape #3}
  \end{tabular}
  \hfill% move it to the right
  \begin{tabular}{l@{}}
     {\itshape #2}
  \end{tabular}
  \ifx&#7&%
  \else{\\%
    \begin{minipage}{\maincolumnwidth}%
      \small#7%
    \end{minipage}}\fi%
  \par\addvspace{#1}}

\newcommand*{\customcvproject}[4][.25em]{
%   \vfill\noindent
  \begin{tabular}{@{}l} 
    {\bfseries #2}
  \end{tabular}
  \hfill% move it to the right
  \begin{tabular}{l@{}}
     {\itshape #3}
  \end{tabular}
  \ifx&#4&%
  \else{\\%
    \begin{minipage}{\maincolumnwidth}%
      \small#4%
    \end{minipage}}\fi%
  \par\addvspace{#1}}

\setlength{\tabcolsep}{12pt}

%----------------------------------------------------------------------------------
%            content
%----------------------------------------------------------------------------------
\begin{document}
%\begin{CJK*}{UTF8}{gbsn}                          % to typeset your resume in Chinese using CJK
%-----       resume       ---------------------------------------------------------
\makecvtitle
\vspace*{-23mm}

\begin{center}
\begin{tabular}{ c c c c }
\faEnvelopeO\enspace alievae0410@gmail.com & \faGithub\enspace \href{https://github.com/alievgithub}{alievgithub} &  \faMobile\enspace +7-964-585-6317

\end{tabular}
\end{center}

\section{EDUCATION}
{\customcventry{2019 - 2023}{B.S., System Analysis and Management (GPA: 4.3/5.0)}{Moscow Institute of Physics and Technology}{Moscow, Russia}{}{}}
{\customcventry{2019 - 2023}{B.S., Economics (GPA: 4.4/5.0)}{The Russian Presidential Academy of National Economy and Public Administration}{Moscow, Russia}{}{}}


\section{PROJECTS AND COURSES}
{\customcvproject{Coursera.org}{2020-2021}
  {\begin{itemize}
    \item C++: The White Belt by Yandex
  \end{itemize}
  }
}
{\customcvproject{\href{https://github.com/alievgithub/music_decomposition}{Decomposing a music track into notes}}{2021}
  {\begin{itemize}
  \item Wrote a Telegram Bot in Python
  \end{itemize}
  }
}


{\customcvproject{Robotics}{2015-2019}
  {\begin{itemize}
    \item Developed a robot for World Robotics Olympiad 2015-2018
    \item Developed a robot for Junior Skills 2017
    \item Developed a robot for RoboCup 2015-2017
  \end{itemize}
  }
}


{\section{SKILLS}

{\customcvproject{Hard}{}
  {\begin{itemize}
    \item Mathematics (mathematical analysis, linear algebra)
    \item Python (Pandas, Scipy, Numpy)
    \item C++
    \item English(B2/C1)
  \end{itemize}
  }
}
{\customcvproject{Soft}{}
  {\begin{itemize}
    \item Purposefulness
    \item Adaptability
    \item Teamwork
  \end{itemize}
  }
}

}

{\section{ACHIEVEMENTS}

{\customcvproject{World Robotics Olympiad}{2015-2018}
  {\begin{itemize}
     \item Moscow stage 2015 \hfill Bronze medal
     \item Moscow stage 2016 \hfill Silver medal and entering the Russian stage
     \item Moscow stage 2018 \hfill Gold medal and entering the Russian stage
  \end{itemize}
  }
}
{\customcvproject{Junior Skills Olympiad}{2017}
  {\begin{itemize}
    \item Winner in Moscow and medalist in Russia in the category of robotics
  \end{itemize}
  }
}

{\customcvproject{RoboCup Olympiad}{2015-2016}
  {\begin{itemize}
    \item Second place in 2015 and 2016 in the sports robotics category
  \end{itemize}
  }
}

{\customcvproject{Gazprom Olympiad}{2019}
  {\begin{itemize}
    \item Second place in mathematics for grade 11
  \end{itemize}
  }
}

{\customcvproject{Phystech Olympiad}{2019}
  {\begin{itemize}
    \item Second place in physics for grade 11
  \end{itemize}
  }
}

{\customcvproject{Sports}{}
  {\begin{itemize}
    \item Kitesurfing and Windsurfing
    \item Tennis
    \item Ski and Snowboard (have a few medals for first and second places)
    \item Aircraft control yak-52
    \item Skydiving
    \item Chess (took prizes in tournaments, a couple of times first place)
    \item Boxing
  \end{itemize}
  }
}

}

% Publications from a BibTeX file without multibib
%  for numerical labels: \renewcommand{\bibliographyitemlabel}{\@biblabel{\arabic{enumiv}}}% CONSIDER MERGING WITH PREAMBLE PART
%  to redefine the heading string ("Publications"): \renewcommand{\refname}{Articles}

% Publications from a BibTeX file using the multibib package
%\section{Publications}
%\nocitebook{book1,book2}
%\bibliographystylebook{plain}
%\bibliographybook{publications}                   % 'publications' is the name of a BibTeX file
%\nocitemisc{misc1,misc2,misc3}
%\bibliographystylemisc{plain}
%\bibliographymisc{publications}                   % 'publications' is the name of a BibTeX file

%-----       letter       ---------------------------------------------------------

\end{document}


%% end of file `template.tex'.
